\documentclass[5p,twocolumn,final]{elsarticle}

%\newcommand{\exerciseno}{1}
%\newcommand{\teamno}{0}


\makeatletter
\def\ps@pprintTitle{%
 \let\@oddhead\@empty
 \let\@evenhead\@empty
 %\def\@oddfoot{\emph{Algorithm Engineering -- Team \teamno{} -- Exercise \exerciseno{}\hfill}}%
 \let\@evenfoot\@oddfoot}
\makeatother
\usepackage{lipsum}
\usepackage{lineno,hyperref}
\usepackage{tikz,pgfplots,pgfplotstable,xcolor}
\usepackage{ifthen}
\modulolinenumbers[5]
\bibliographystyle{elsarticle-num}


\pgfplotsset{
	discard if not/.style 2 args={
		x filter/.code={
			\edef\tempa{\thisrow{#1}}
			\edef\tempb{#2}
			\ifx\tempa\tempb
			\else
				\def\pgfmathresult{inf}
			\fi
		}
	},
}
\pgfmathdeclarefunction{lg2}{1}{%
    \pgfmathparse{ln(#1)/ln(2)}%
}
\pgfmathdeclarefunction{lg10}{1}{%
    \pgfmathparse{ln(#1)/ln(10)}%
}

%%%%%%%%%%%%%%%%%%%%%%%

\begin{document}
\begin{frontmatter}
	\title{Algorithm Engineering -- Exercise X}
	\author{Team Y: Grace Hopper, Ada Lovelace, and Joseph Weizenbaum}
%\begin{keyword}
%	Add keywords here
%\end{keyword}
\end{frontmatter}

\linenumbers
\section*{Remarks [Not to be contained in the final version]}
\begin{itemize}
	\item Write a bit to each of the following four sections (at most one page of text, including references!)
	\item Put the figures (at least one) on the subsequent pages.
	\item Use references when appropriate. E.\,g.\ always reference Skiena \cite{skiena1998} ;-)
\end{itemize}

\section{Implemented Features}
Introduce the method(s) employed in your submission to the exercise.
Ideas explained in the lecture can just be mentioned and do not need to be explained (again).

\section{Data Structures}
Tell us something about the data structures you used, and why they were a good (or not so good) choice.

\section{Highlights}
You have some freedom here.
Tell us about the most interesting observations you made during working on the assignment!

\section{Experiments}
You should do a few experiments on your submission, showing how well your solver performs.
For example (as a bare minimum), compare it to the solver of your previous submission.
Tell us here what experiments you made.
Use some figures to convince us of how great your submitted solver is.
See Figure~\ref{fig:example-diagrams} for two example diagrams.
The plots are built from the data supplied in \texttt{data.csv}.
For the top plot, the CSV needs to contain the running times of the two algorithms to compare.
For the bottom plot, the CSV needs to contain the running time of one algorithm, the graph size, and the solution size.
Thus, for the bottom plot you can easily use the CSV generated by the script \texttt{benchmark-fast.sh}, while for the top plot, you first have to merge the CSV files of two benchmark runs (one for each algorithm).
Observe that you can load multiple CSV files into the plots and, for example, give them different symbols.

% \section*{References}

\bibliography{bib}

%\clearpage
\onecolumn

\begin{figure}[t]
	\centering
	\def\maxValue{500} %
	\def\minValue{0.1} % choose such that all values are displayed
	\begin{tikzpicture}[scale=1]
		\begin{loglogaxis}[
					width=0.9\textwidth,
					height=0.55\textwidth,
					title={This is a plot for comparing running times},
					xlabel={Algorithm A [s]},
					ylabel={Algorithm B [s]},
					xmax=\maxValue,
					xmin=\minValue,
					ymax=\maxValue,
					ymin=\minValue,
					legend cell align=left,
					legend style = {
						at={(0.02, 0.98)}, % where is the legened located?
						anchor={north west}, % what corner of the legend box is specified in line above?
% 						font = \small  % if there is not enough space...
					}
				]
			
			% the data points from the csv files; here if one of the two entries in the csv-file is empty, then the value is replaced by 300:
			\addplot[
				only marks,color=black,mark=+,thick,
				x filter/.code={\ifthenelse{\equal{\pgfmathresult}{}}{\def\pgfmathresult{5.703782475}}{}}, % ln(300) = 5.703782475... ; needed due to logarithmic scale
				y filter/.code={\ifthenelse{\equal{\pgfmathresult}{}}{\def\pgfmathresult{5.703782475}}{}}  % ln(300) = 5.703782475... ; needed due to logarithmic scale
				] table [col sep=semicolon,x={Atime}, y={Btime}] {compare.csv};
			\addlegendentry{Test Instances}

			% helper lines indicating factor 1, 5, and 25:
			\addplot[color=black,domain=\minValue:\maxValue,samples=2] {x};
			\addlegendentry{Factor 1}
			\addplot[dashed,color=black!75,domain=\minValue:\maxValue,samples=2] {5*x};
			\addlegendentry{Factor 5}
			\addplot[dotted,color=black,domain=\minValue:\maxValue,samples=2] {25*x};
			\addlegendentry{Factor 25}

			\addplot +[color=red,solid,mark=none] coordinates {(\minValue, 300) (\maxValue, 300)};
			\addlegendentry{Timeout}

			% add plots without legend entry last:
			\addplot +[color=red,solid,mark=none] coordinates {(300, \minValue) (300, \maxValue)};
			\addplot[dashed,color=black!75,domain=\minValue:\maxValue,samples=2] {0.2*x};
			\addplot[dotted,color=black,domain=\minValue:\maxValue,samples=2] {0.04*x};
			
		\end{loglogaxis}
	\end{tikzpicture}
	
	\medskip
	
	\begin{tikzpicture}
		\begin{axis}[
				width=0.8\textwidth,
				height=0.6\textwidth,
				grid,
				xmode=log,
				ymode=log,
				title={A fancy plot with colors},
				xlabel={$n$},
				ylabel={$k$},
				legend cell align=left,
				legend style = {
					at={(0.02, 0.98)}, % where is the legened located?
					anchor={north west}, % what corner of the legend box is specified in line above?
					font = \small  % if there is not enough space...
				},
				colorbar,
				colorbar style={
					ytick={-1,0,1,2},
					ylabel=time in seconds,
					ylabel style={
						yshift=-7em
					},
					yticklabel={$10^{\pgfmathprintnumber{\tick}}$},
				},
			]
			\addplot[only marks,scatter src=explicit, scatter,mark=+,thick,discard if not={finished}{1}] table[col sep=semicolon,y={solsize},x={vertices},meta expr=lg10(\thisrow{time} + 0.01)] {results-2020-04-16-17-01-55-1-random.csv};
			\addlegendentry{Random}
			\addplot[only marks,scatter src=explicit, scatter,mark=o,thick,discard if not={finished}{1}] table[col sep=semicolon,y={solsize},x={vertices},meta expr=lg10(\thisrow{time} + 0.01)] {results-2020-04-16-17-01-55-2-social-networks.csv};
			\addlegendentry{Social Networks}
			\addplot[only marks,scatter src=explicit, scatter,mark=square,thick,discard if not={finished}{1}] table[col sep=semicolon,y={solsize},x={vertices},meta expr=lg10(\thisrow{time} + 0.01)] {results-2020-04-16-17-01-55-3-medium-sized.csv};
			\addlegendentry{Medium}
		\end{axis}
	\end{tikzpicture}
	\caption{
		\textbf{Top:} Comparison of the running times of two algorithms.
		The solid line denotes the values where both algorithms would have the same running time.
		Marks above the solid line indicate an instance where Algorithm A is faster, marks below the solid line indicate an instance where Algorithm B is faster.
		The dashed lines mark the factor 5 increase/decrease and the dotted lines mark the factor 20 increase/decrease. 
		\textbf{Bottom:} An example for a heat map showing the running time depending on~$k$ and~$n$ at the same time.
		The marks indicate random instances (mark +), social networks (circles), and medium-sized graphs (squares).
		All scales are log-scales.
	}
	\label{fig:example-diagrams}
\end{figure}




\end{document}
