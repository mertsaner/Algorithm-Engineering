%----------------------------------------------------------------------------------------
%	PACKAGES AND THEMES
%----------------------------------------------------------------------------------------

\documentclass{beamer}

\mode<presentation> {

% The Beamer class comes with a number of default slide themes
% which change the colors and layouts of slides. Below this is a list
% of all the themes, uncomment each in turn to see what they look like.

    %\usetheme{default}     %ordinary
    %\usetheme{Boadilla}  %nice
     \usetheme{Madrid}  %very nice


% As well as themes, the Beamer class has a number of color themes
% for any slide theme. Uncomment each of these in turn to see how it
% changes the colors of your current slide theme.

%\usecolortheme{beaver}   %red
%\usecolortheme{crane}    %yellow
%\usecolortheme{dolphin}    %blue, nice  and same with \usetheme{Boadilla}
%\usecolortheme{dove}			%black and white
%\usecolortheme{lily}					%normal
%\usecolortheme{orchid}				%normal
%\usecolortheme{rose} 					%graphs are open blue
%\usecolortheme{seagull}					%colorless
%\usecolortheme{seahorse}				%nice turkuaz color
%\usecolortheme{whale}					%no change
%\usecolortheme{wolverine}				%yellow-blue

%\setbeamertemplate{footline} % To remove the footer line in all slides uncomment this line
%\setbeamertemplate{footline}[page number] % To replace the footer line in all slides with a simple slide count uncomment this line

%\setbeamertemplate{navigation symbols}{} % To remove the navigation symbols from the bottom of all slides uncomment this line
}

\usepackage{graphicx} % Allows including images
\usepackage{booktabs} % Allows the use of \toprule, \midrule and \bottomrule in tables

%----------------------------------------------------------------------------------------
%	TITLE PAGE
%----------------------------------------------------------------------------------------
\title[Exercise 2: Weighted Cluster Editing II]{Algorithm Engineering Exercise 2 } % The short title appears at the bottom of every slide, the full title is only on the title page
\subtitle {Weighted Cluster Editing II}

\author[Saner,  Bähnisch,  Wetterau]{\textbf{Team 1 } \newline Mert Saner \newline Max Bähnisch \newline Hannah Wetterau  \newline} 
 \date{ \today }


\institute[] % Your institution as it will appear on the bottom of every slide, may be shorthand to save space
{
\textit{TU Berlin} \\\textit{Institute of Software Engineering and Theoretical Computer Science} \\ % Your institution for the title page
\medskip
%\textit{bofu20131@163.com} % Your email address
}


\begin{document}
%\begin{frame}  %for section part uncomment it
\titlepage % Print the title page as the first slide
%\end{frame}

%\begin{frame}
%\frametitle{Overview} % Table of contents slide, comment this block out to remove it
\tableofcontents % Throughout your presentation, if you choose to use \section{} and \subsection{} commands, these will automatically be printed on this slide as an overview of your presentation
%\end{frame}

%----------------------------------------------------------------------------------------
%	PRESENTATION SLIDES
%----------------------------------------------------------------------------------------

%------------------------------------------------
%\section{First part ?} % Sections can be created in order to organize your presentation into discrete blocks, all sections and subsections are automatically printed in the table of contents as an overview of the talk
%------------------------------------------------
%\section{Second Part/question ?}
%\section{ Third .}
%\section{Last }

\begin{frame}
\frametitle{Intro}

1. Problem

\begin{itemize}
\item  Implementation
\item  Implementation
\item  Implementation
\end{itemize}
\end{frame}

%------------------------------------------------
\begin{frame}
\frametitle{Our Results}

1. Results

\begin{itemize}
\item  Implementation Steps
\item  Implementation Steps
\item  Implementation Steps
\end{itemize}
\end{frame}

%------------------------------------------------

\begin{frame}
\frametitle{Experiment I }

2. Add Graph

\begin{itemize}
\item  Points
\item Points
\item Points
\end{itemize}
\end{frame}

%------------------------------------------------

\begin{frame}
\frametitle{Experiment II}

3. Third point 
\begin{itemize}
\item  Points
\item Points
\item Points
\end{itemize}
\end{frame}

%------------------------------------------------

\begin{frame}
\frametitle{Summary and Outlook }

3. Key Results:
\begin{itemize}
\item First Point
\item Second Point
\end{itemize}
\end{frame}

%------------------------------------------------

\begin{frame}
\frametitle{Heading }
\begin{block}{Step 1: SubHeading }

\end{block}

\begin{block}{Step 2: SubHeading}

\end{block}

\begin{block}{Step 3: SubHeading}

\end{block}
\end{frame}

%------------------------------------------------
\begin{frame}
\frametitle{Heading }

\begin{block}{Step : SubHeading}
Sentence- Description
\end{block}
\end{frame}

%------------------------------------------------



\begin{frame}
\frametitle{Heading/Figure}
Sentence/Figure
\begin{figure}
%\includegraphics[width=0.8\linewidth]{vw.png}
  \caption{Name of the Figure }\label{fig:digit}
\end{figure}
\end{frame}

%------------------------------------------------

\begin{frame}
\frametitle{Last Heading }

4. Last Heading

\begin{itemize}
\item Last Part
\end{itemize}
\end{frame}

%------------------------------------------------

\begin{frame}

%\Huge{\centerline{Thank you for listening!}}
\begin{center}
\includegraphics[width=0.8\linewidth]{picture.jpeg}

\end{center}
%\Huge{\begin{center}
 %The End \newline Thank You for Listening 
%\end{center}}

\end{frame}

%----------------------------------------------------------------------------------------
\end{document} 